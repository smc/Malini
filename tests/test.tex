\documentclass{article}

\usepackage{xcolor}
\usepackage{fontspec}
\usepackage{polyglossia}

\setmainfont[
    Extension=.ttf,
    % Colour=black,
    UprightFont=*,
    Renderer=HarfBuzz,
    UprightFont=*,
    UprightFeatures={RawFeature={+axis={wght=400,wdth=100}}},
    BoldFont=*,
    BoldFeatures={RawFeature={+axis={wght=700,wdth=105}}},
]{Malini-VF}

% \setmainfont[Renderer=HarfBuzz]{Manjari}

\defaultfontfeatures{HyphenChar={"00AD}}

\setdefaultlanguage{malayalam}
\setlanghyphenmins{malayalam}{3}{4}
\linespread{1.2}
\widowpenalty=10000
\clubpenalty=10000
\raggedbottom
\sloppy
\begin{document}
This is the main font (default).

% \maliniexamplefonthb
പെരിയോനെ എൻ റഹ്‌മാനെ, പെരിയോനെ റഹീം.

മലയാളികൾക്ക് സ്വന്തമായില്ലാത്ത ഒരു ഴോണറിലുള്ള ഗാനം. എ ആർ റഹ്മാൻ, താൻ ഇതുവരെ ഉപയോഗിക്കാത്ത രാഗത്തിൽ നെയ്തെടുത്ത ഗാനം. പെരിയോനെ എന്നെഴുതിയത് റഹ്മാൻ തന്നെയാണ്. എന്താണ് പെരിയോൻ? ആരാണ് പെരിയോൻ?

മലയാളിക്കറിയാത്ത ഒരാളല്ല മസറയിൽ പെട്ടു പോയ നജീബ്. മൂന്ന് കൊല്ലം, നാല് മാസം, പന്ത്രണ്ട് ദിവസം പ്രത്യാശയേതുമില്ലാഞ്ഞിട്ടും ആടുജീവിതം ജീവിച്ച നജീബ്. ബ്ലെസ്സിയോട് താൻ ഒരുപാട് തവണ തർക്കിച്ചത് നജീബ് എന്തേ ആത്മഹത്യക്ക് ശ്രമിച്ചില്ല എന്നതാണ് എന്ന് എ ആർ റഹ്‌മാനോട് പൃഥ്വിരാജ് പറയുമ്പോൾ, എ ആർ റഹ്മാൻ അതിന് കൊടുക്കുന്ന മറുപടി, മലയാളി കേട്ട് കാണും. 'ഇസ്‌ലാമിൽ ആത്മഹത്യ ഹറാമാണ്. അന്ന് വരേക്കും ചെയ്ത നന്മയും, ത്യാഗവും, കാത്തുസൂക്ഷിച്ച വിശ്വാസവും ആത്മഹത്യ ചെയ്യും വഴി വിശ്വാസിയിൽ നിന്ന് അകറ്റപ്പെടും എന്നൊരു വിശ്വാസമുണ്ട്. വിശ്വാസമാണ് നജീബിനെ നയിച്ചത് എന്നാണ് റഹ്മാൻ പറയുന്നത്. അത് പറഞ്ഞ മറ്റൊരാൾ കൂടിയുണ്ട്. ബെന്യാമിൻ. യഥാർത്ഥ നജീബ്, വിശ്വാസിയായിരുന്നില്ല. അയാൾ മരുഭൂമിയിൽ വച്ച് പലകുറി ആത്മഹത്യയ്ക്ക് ശ്രമിച്ചിട്ടുണ്ട്. ആടുജീവിതത്തിന്റെ ആദ്യ കോപ്പികളിൽ ബെന്യാമിൻ ഒരു എപ്പിലോഗ് എഴുതിയിരുന്നു. ആടുജീവിതം ഒരു ജീവചരിത്രമല്ല എന്നും, അതൊരു നോവൽ ആണെന്നും പറഞ്ഞുകൊണ്ട്. നജീബിന്റെ ജീവിതവും, ബെന്യാമിന്റെ വിശ്വാസവും ചേർന്നതാണ് ആടുജീവിതം എന്ന നോവൽ എന്ന്. അതിനെ നീട്ടി വായിച്ചാൽ നജീബിന്റെ ജീവിതവും, റഹ്മാൻറെ വിശ്വാസവും ചേർന്നതാണ് പെരിയോനെ എന്ന് പറയാം. പെരിയോനെ എന്നെഴുതിയത് റഫീഖ് അഹമ്മദ് അല്ല. എ ആർ റഹ്മാൻ തന്നെയാണ്. ആ സംഗീതത്തിന് ആത്മീയമായ ഒരു തലം അന്വേഷിച്ച്, ദൈവത്തിന് നിങ്ങൾ എന്ത് പറയും എന്ന് ചോദിച്ച്, റഹ്മാൻ കണ്ടെത്തിയ ഉത്തരം. പെരിയോനെ.

ഗാനത്തിൻറെ സംഗീത ഘടന, മലയാള ചലച്ചിത്രഗാന ശാഖയിൽ ഇല്ലാതിരുന്നത് എന്ന് തന്നെ പറയാം. കുൻ ഫയ കുൻ ആയിരിക്കും പെട്ടന്ന് ശ്രോതാവിന് പെരിയോനോട് ചേർന്നു നിൽക്കുന്ന ഒരു ഗാനമായി ഓർമ്മ വരുന്നത്. ആത്മീയമായ ഒരു ട്രിപ്പ് തരുന്ന സംഗീതം. അതിന്റെ മ്യൂസിക് ഒൺലിയുടെ പേര് ബിനോവേലന്റ് ബ്രീസ് എന്നാണ്. കരുണാമയമായ കാറ്റ്. മരുഭൂമിയിലെ പൊടിക്കാറ്റിന് തീർച്ചയായും കരുണയില്ല. പിന്നെന്താണ് ബിനോവേലന്റ് ബ്രീസ്?

പെരിയോനെ പാട്ടിന്റെ വീഡിയോ തുടങ്ങുന്നത്, മരുഭൂമിയിൽ വീണു കിടക്കുന്ന ഒരു സോഡാക്കുപ്പിയെടുത്ത് അതിൽ നിന്ന് സംഗീതമുരുവാക്കുന്ന എ ആർ റഹ്‌മാനിൽ നിന്നാണ്. ദയാദാക്ഷിണ്യമില്ലാത്ത ആരോ ഉപേക്ഷിച്ചു പോയ ഒരു വസ്തുവിൽ നിന്ന് സംഗീതമുരുവാക്കുന്ന സംഗീതജ്‌ഞൻ. അത് പ്രതീക്ഷയാണ്. മരുഭൂമിയിലെ മരുപ്പച്ച. ബ്ലെസ്സി എന്ന ചലച്ചിത്രകാരൻ അത് വെറുതെ എടുത്തതാകാൻ വഴിയില്ല. പാട്ടിലുടനീളം കാറ്റിന്റെ ആ സാന്നിധ്യം കേട്ടറിയാം. ഹമ്മിങ് ആയും, കൊറസ് ആയും വാദ്യോപകരണങ്ങൾ ആയും കാറ്റ് പാട്ടിലുണ്ട്. കാറ്റ് നിന്നു പോകുന്ന ഒരു ചെറിയ ഇടവേള ബ്ലെസ്സി നിറക്കുന്നത്, നജീബിന്റെ 'ഉമ്മാ' എന്ന വിളിയാലാണ്. കാറ്റടങ്ങിയ മരുഭൂമിയിൽ അസ്തമയ സൂര്യനെ നോക്കി, ഉമ്മാ എന്ന് നീട്ടി വിളിക്കുന്ന നജീബ്.

XeTeX is packaged for all famous GNU/Linux distros. The installation method depends your distro. For ease of installation and configuration, we suggest to use a TeXLive version 2012 or above – either standalone TeXLive distribution or install from your distribution’s package manager. Windows and OSX versions are also available.

Following packages are required to install to get a working xetex environment in your computer. Note that these packages are relatively large in size and will take time and bandwidth.

\end{document}
